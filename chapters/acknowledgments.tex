\addchap{\lsAcknowledgementTitle} 
\markdouble{\lsAcknowledgementTitle}
Das vorliegende Buch habe ich an den Universitäten Hamburg und Bamberg als Dissertation verfasst und für die Veröffentlichung leicht überarbeitet. 
Eine solche Arbeit entsteht selbstverständlich nicht ohne viel Unterstützung und Rückhalt -- dafür möchte ich mich ganz herzlich bedanken. 

Mein erster Dank gilt meiner Doktormutter Renata Szczepaniak, die mich für sprachliche Variation und damit für eine Promotion in der Linguistik begeistert hat. 
Von ihr habe ich nicht nur gelernt, Sprache durch eine ganz neue Brille zu betrachten, sondern sie hat mir auch sehr viel über gutes wissenschaftliches Schreiben und konstruktiven wissenschaftlichen Diskurs beigebracht.
Jürgen Spitzmüller, der die Dissertation zweitbetreut hat, danke ich für den wertvollen Austausch, die guten Anregungen und die sehr inspirierende Summer School in Wien mit Michael Silverstein. 

Meine Promotionszeit kann ich mir nicht ohne Eleonore Schmitt und Lisa Dücker vorstellen. Sie waren bei allen Aufs und Abs dabei, haben mit mir gemeinsame Tagungen, Mittagspausen und Bürostunden verbracht und mir in Bamberg ein Dach über dem Kopf gegeben. 
Florian Busch danke ich für die meta-metapragmatischen Unterhaltungen und wertvolle inhaltliche Anmerkungen. 
Ich danke Stefan Hartmann für die stets sekundenschnelle Lösung jeglicher R-Probleme.
Johanna Flick und Melitta Gillmann danke ich für gute Ratschläge und ein offenes Ohr. 
Für Struktur in der Corona-Pandemie und Pancakes im Wendland danke ich Melanie Andresen, Alan van Beek, Lisa Merten und Malena Ratzke. 
Und auch die anderen Kolleg:innen der Hamburger und der Bamberger Germanistik haben dazu beigetragen, dass ich die Promotionszeit immer in guter Erinnerung behalten werde. 
Ein großer Dank und ein kleines Hinterherweinen gebührt außerdem dem Café Creisch -- dieser Ort und alle, die dazu gehört haben, haben mein Studium und meine Promotionszeit sehr geprägt. 

Ganz wichtig zu erwähnen sind nicht zuletzt alle, die an der Befragung teilgenommen und damit die Auswertung und Analyse erst ermöglicht haben. 
Für die Unterstützung bei der Kodierung der Daten danke ich Josephine Hinrichs und Marie Wrona. 
Des Weiteren danke ich allen, die Teile meiner Dissertation korrekturgelesen haben, in welchem Stadium auch immer, etwa Anna Kayser, Friedemann Bretschneider und Helen Kahlert.
Felix Kopecky und Sebastian Nordhoff von Language Science Press, die Herausgeber:innen von Open Germanic Linguistics sowie die Reviewer:innen haben dazu beigetragen, dass das Buch den nötigen Feinschliff erfahren hat. 

Sicher wäre der Weg zu dieser Arbeit anders verlaufen ohne die Unterstützung und das Vertrauen meiner Familie. 
Ganz besonders möchte ich Marco danken für die Geduld, das Verständnis und fürs Rückenfreihalten. 