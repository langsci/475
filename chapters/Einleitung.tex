\chapter{Einleitung}
Präpositionen wie \wegen{} oder \dank{} variieren im Sprachgebrauch zwischen der Genitivrektion und der Dativrektion:\footnote{Da die Voranstellung im Deutschen am häufigsten auftritt, wird die Bezeichnung \object{Präposition} hier als Oberbegriff für prä-, post- und zirkumponierte Adpositionen gewählt, wie auch in anderen Arbeiten üblich (\citealp[vgl. etwa][629--630]{Griehaber2009}; \citealp[39]{Romare.2004}; \citealp[356]{Helbig.2017}).} 
\begin{exe}
\ex \object{Glücklicherweise entscheiden sich Menschen nicht nur \textbf{wegen des Geldes} für ein Studium, sondern aus Neigung.} (DWDS, 2017, Die Zeit)
\ex \object{Der Grundwasserspiegel ist \textbf{wegen dem Hochwasser} stark angestiegen.} (DWDS, 2002, Der Tagesspiegel)
\end{exe}
Diese Variation wird von SprachbenutzerInnen seit Langem reflektiert, diskutiert und bewertet. 
So erhalten etwa Sprachberatungsstellen wie das Essener Sprachtelefon zahlreiche Anfragen hierzu \citep[s.][121]{Bunting1996}. 
Ein prominentes Beispiel aus dem Diskurs um die Kasusrektion von Präpositionen sind zudem die Veröffentlichungen Bastian Sicks unter dem Titel \textit{Der Dativ ist dem Genitiv sein Tod}. 
In diesem Titel zeigt sich bereits, dass der Dativ gegenüber dem Genitiv stigmatisiert wird. 
Diese sprachideologische Aufladung wird auch in folgendem humoristischen Spruch deutlich, den die Plattform iamstudent.de über Instagram geteilt hat:
\begin{exe}
\ex \object{Alle Studis jammern wegen dem Lernen. Außer die Germanistikstudis. Die jammern wegen des Lernens.} (URL: \url{https://www.instagram.com/iamstudent_at p/Bqo0XiZjwtp}, zuletzt aufgerufen am 26.08.2020)
\end{exe}
%\begin{figure}[H]
%\centering
%\includegraphics[scale=0.085]{TwitterStudis.jpg}
%\caption{Instagram-Post der Plattform iamstudent.de}
%\label{pic:studiinsta}
%\end{figure}
Die Genitivrektion mit \wegen{} wird hier den Germanistikstudierenden zugeschrieben, die als besonders kompetent in der Grammatik des Deutschen gelten. 
Sie steht somit für eine hohe sprachliche Sicherheit und damit verbundene Werte wie hohe Bildung. 
Die Kasus erfahren also eine sozialsymbolische Aufladung und werden daher von den SprachbenutzerInnen als indexikalische Verweise auf soziale Kategorien gedeutet \citep[s.][]{Silverstein2003}.

Die zentrale Hypothese der vorliegenden Studie ist, dass diese Indexikalität von Dativ und Genitiv für die Variation % Änderung Anfangund den Wandel % Änderung Ende 
der präpositionalen Rektion entscheidend ist. 
Diese Integration der Bewertung der Präpositionalkasus ist im Vergleich zu vorherigen Studien zur Variation der Rektion von Präpositionen neu.
Für einen umfassenden Blick auf das Phänomen ist sie jedoch relevant, denn bei der Erforschung sprachlicher Variation sollten drei Perspektiven berücksichtigt werden, wie \citet[223]{Silverstein.1985} betont: 
\begin{enumerate}
\item die systemlinguistische Perspektive (welche grammatischen Regularit{\"a}ten sind erkennbar?), 
\item die variationslinguistische Perspektive (welche gruppenspezifischen Verteilungsmuster sind erkennbar?) und 
\item die sprachideologische Perspektive (wie werden die Varianten in der Sprachgemeinschaft oder in bestimmten Gruppen bewertet? Wie wird das Ph{\"a}nomen konzeptualisiert?).
\end{enumerate} 
Diese drei Bereiche hängen untrennbar miteinander zusammen: 
\begin{quote}[A]ll three elements -- linguistic form, social use, and human reflections on these forms in use -- mutually shape and inform each other. To understand and explain any of them we must take into account both of the other two, in Silverstein's view.~\citep[436]{Woolard2008}\end{quote}
%\citet[3]{Cuonz.2014} etwa geht davon aus, dass aufgrund der Komplexit{\"a}t metapragmatischer Bewertungen bei deren Beschreibung und Analyse verschiedene Konzepte kombiniert werden sollten.
%\begin{quote}
%All attitudes to language and language change are fundamentally ideological, and the relationship between popular and expert ideologies, though it is complex and conflictual, is closer than one might think. \citep[4]{Cameron1995}
%\end{quote}
%So kommt natürlich auch mein Interesse an der Kasusrektion der Sekundärpräpositionen und ihrer metapragmatischen Wertung nicht aus dem Nichts, sondern gründet sich auf die konstruktivistische Überzeugung, dass dieser Wandel nicht ohne die Betrachtung der damit verbundenen Sprachideologien erklärt werden kann. \\
Dennoch wurden die Schwankungen in der präpositionalen Rektion in der Linguistik bisher überwiegend im Rahmen der Grammatikalisierungstheorie untersucht, die sich mit der Entstehung und Festigung grammatischer Strukturen beschäftigt (\citeauthor[s. vor allem die Arbeiten von][]{DiMeola2000}, etwa \citeyear{DiMeola2000}, \citeyear{DiMeola2003} und \citeyear{DiMeola2006}). 
Die Variation zwischen Genitiv- und Dativrektion wird hier als Effekt zweier verschiedener Prinzipien im Grammatikalisierungsprozess gesehen, der Prototypisierung und der Differenzierung. 
Mit Prototypisierung ist gemeint, dass Präpositionen im Laufe ihrer Grammatikalisierung die Eigenschaften typischer, also häufig verwendeter und fest im grammatischen System verankerter Präpositionen des Deutschen, wie etwa \object{mit} oder \object{zu},  annehmen~\citep[s.][17]{Lindqvist1994}. 
Eine dieser Eigenschaften ist die Dativrektion~(\citealp[s.][15--16]{Lindqvist1994}; \citealp[94]{Szczepaniak2011}). 
Die Prototypisierung hat demnach zur Folge, dass Präpositionen, die ursprünglich den Genitiv regieren, nach und nach zur Dativrektion übergehen.
Präpositionen, die ursprünglich den Dativ regieren, wie bspw. \dank{}, entsprechen in diesem Punkt bereits dem Prototyp, weshalb vermutet werden könnte, dass sie den Rektionskasus beibehalten.
Es lässt sich aber beobachten, dass viele von ihnen zur Genitivrektion wechseln \citep[s.][256]{DiMeola2005b}. 
So wird etwa \dank{} mittlerweile überwiegend mit dem Genitiv gebraucht \citep[s.][§915]{Duden2016}. 
Die Entwicklung vom Dativ zum Genitiv wird grammatikalisierungstheoretisch mit dem Prinzip der Differenzierung erklärt. 
Unter Differenzierung wird verstanden, dass sich Präpositionen formal von ihrer Spenderstruktur abgrenzen (\citealp[s.][144]{DiMeola2000}; \citealp[422]{DiMeola2006}): 
Im Falle von \dank{} trägt der Genitiv dazu bei, dass das Wort nicht mehr als Substantiv (wie in \object{Dank sei dem X}) gelesen werden kann, sondern eindeutig als Präposition erkennbar ist (wie in \object{dank des X}).

Die Grammatikalisierungstheorie und die Prozesse von Prototypisierung und Differenzierung bieten jedoch für zwei Beobachtungen keine ausreichende Erklärung: 
Erstens wird die Genitivrektion offenbar von einigen Präpositionen, wie etwa \wegen{} oder \waehrend{}, erstaunlich lange beibehalten, auch wenn ihre Grammatikalisierung bereits vorangeschritten ist (\citealp[s.][218--219]{DiMeola2003}; \citealp[][214]{Vieregge.2019}).
So gibt der \citet[§915]{Duden2016} für \waehrend{} mit einer Nominalphrase im Singular 9~\% Dativ an und für \wegen{} gerade einmal 1~\%. 
Zweitens findet der Wechsel zum Genitiv teilweise deutlich schneller statt als der Wechsel zum prototypischen Dativ (\citealp[s.][216]{DiMeola2000}; \citealp[][214]{Vieregge.2019}). 
Dies ist etwa bei \dank{} der Fall \citep[s.][257]{Baumann2014}. 
Hier stellt der \citet[§915]{Duden2016} 73~\% Genitiv bei Nominalphrasen im Singular fest. 
Neben der Grammatikalisierung der Präpositionen muss also ein weiterer wesentlicher Faktor die Variation beeinflussen. 

Daher ist die Berücksichtigung der sozialen Bewertung von Dativ und Genitiv entscheidend, denn wie \citet[45]{Preston2004} ausführt, wird der Gebrauch sprachlicher Formen von ihrer Bewertung beeinflusst: 
\begin{quote}It is hardly surprising, therefore, to find that finely-tuned choices among linguistic features, reflecting the social forces and groups which surround them, play as complex a role in attitudinal formation and perception as they do in language variation itself. In fact, it seems to me that perception, evaluation, and production are intimately connected in language variation and change and that much that might go by the name \glqq sociolinguistics\grqq{} could as well be known as \glqq language attitude study\grqq.~\citep[45]{Preston2004}\end{quote}
Die Erhebung von Werturteilen ist für die Erforschung eines Variationsphänomens also zentral. 
Dennoch wurde die systematische Untersuchung des Zusammenhangs zwischen der Bewertung und der Variation einzelner grammatischer Merkmale in der deutschsprachigen Forschung bisher oft vernachlässigt. %\citep[s. aber][zur Stigmatisierung des \object{am}-Progressiv]{}. % Änderung Anfang
Mit der Untersuchung des Zusammenspiels von Bewertung und Variation bei der Rektion von Präpositionen des Deutschen unterstreicht die vorliegende Studie, welch großen Nutzen der Einbezug der Bewertungskomponente für die Erforschung von grammatischen Variationsphänomenen hat. % Änderung Ende
Da es sich bei den Präpositionalkasus um Varianten handelt, die in der sprachlichen Öffentlichkeit sehr prominent diskutiert und sehr wertend behandelt werden, eignen sie sich gut für eine solche Untersuchung.  
Bisher wird der Genitiv in der Linguistik oft als \glqq Prestigekasus\grqq{} bezeichnet (\citealp[s. etwa][36]{Lehmann1992}; \citealp[6]{Zimmer.2018}). 
Jedoch wurden die sozialen Bedeutungen, die SprachbenutzerInnen dem Genitiv und dem Dativ als Rektionskasus zuschreiben, noch nicht systematisch untersucht. 
Die vorliegende Studie hat deshalb zum Ziel, Folgendes zu zeigen: 
\begin{enumerate}
\item Die Genitivrektion und die Dativrektion sind mit je unterschiedlichen Sets sozialer Bedeutungen verknüpft. 
\item Die Rektionsvarianten werden von SprachbenutzerInnen entsprechend ihrer sozialen Bedeutungen je nach Kontext unterschiedlich verwendet. 
\end{enumerate}
Hierzu ist es erforderlich, sowohl explizite Werturteile von SprachbenutzerInnen als auch Gebrauchsdaten zu den beiden Präpositionalkasus zu erheben, sodass die Konzeptualisierung der Varianten mit ihrer Verwendung in unterschiedlichen Situationen abgeglichen werden kann. 
Mithilfe eines Onlinefragebogens wurden für die vorliegende Untersuchung daher knapp 400 Deutsch\hyp MuttersprachlerInnen aus Deutschland befragt.  
Die Bewertung der Genitiv- und Dativrektion wurde in Form von freien Assoziationen und Akzeptabilitätsurteilen abgefragt. 
Um Daten zur Verwendung der Varianten zu erhalten, wurden die Befragten um das Ausfüllen von Lückentexten gebeten. 

Die Präpositionen, die für die Untersuchung ausgewählt wurden, sind \wegen{}, \waehrend{}, \dank{}, \gegenueber{} und \object{seit}. 
Bei \wegen{} und \waehrend{} handelt es sich um ursprünglich den Genitiv regierende Präpositionen, die in ihrer Grammatikalisierung bereits fortgeschritten sind und laut \citet[][§915]{Duden2016} Kasusschwankungen aufweisen. 
\object{Dank} und \gegenueber{} hingegen regieren ursprünglich den Dativ und variieren unterschiedlich stark in ihrer Kasusrektion: 
Während \dank{} mit beiden Kasus vorkommt, ist \gegenueber{} beinahe ausschließlich mit dem Dativ belegt (\citealp[][§915]{Duden2016}). 
Die Präposition \object{seit} gehört zu den hochfrequenten, prototypischen Dativpräpositionen des Deutschen und zeigt daher keine Variation in ihrer Rektion \citep[s.][94]{Szczepaniak2011}. 
Sie wurde ausgewählt, um zu überprüfen, ob die sozialsymbolische Aufladung der Rektionskasus die Genitivrektion auch bei an sich nicht schwankenden Präpositionen hervorrufen kann. 

Das Buch gliedert sich in zwei theoretische, zwei empirische und zwei zusammenfassende Kapitel, deren Aufbau im Folgenden genauer erläutert wird. 
\autoref{cha:SprachideologienundSpracheinstellungen} behandelt den Einfluss der Konzeptualisierung von Sprache auf ihre Variation und ihren Wandel. 
Dabei geht \autoref{sec:MetapragmatikinderLinguistik} zunächst auf die Relevanz der Reflexion über Sprache für die Erforschung linguistischer Phänomene ein. 
Anschließend werden mit der Spracheinstellungsforschung und der Sprachideologieforschung die beiden Forschungsstränge der Linguistik vorgestellt, die sich eingehend mit der Bewertung sprachlicher Formen beschäftigen (\autoref{sec: Forschungsansaetze}):   
Während die Spracheinstellungsforschung vor allem darauf abzielt, persönliche Meinungen zu Varietäten und Varianten zu erheben, ist die Sprachideologieforschung an der Konstruktion von Zusammenhängen zwischen Sprache und Gesellschaft interessiert. 
Dieser Unterschied zwischen den Ansätzen wird zunächst herausgearbeitet. 
Anschließend erfolgt eine theoretische und methodologische Zusammenführung von Spracheinstellungs- und Sprachideologieforschung, um Konzepte und Herangehensweisen beider Traditionen für die vorliegende Untersuchung nutzbar zu machen. 
Das erste Theoriekapitel schließt mit Überlegungen zum Zusammenhang zwischen der Beurteilung und der Verwendung von Sprache, die eine wesentliche theoretische Grundlage der Untersuchung bilden (\autoref{sec:MetapragmatikVariationWandel}). 

\autoref{cha:SekPraeps} ist der bisherigen Forschung zu Variation und Wandel der Rektion von Präpositionen gewidmet. 
Dabei wird zunächst auf die Eigenschaften prototypischer und peripherer Vertreter der Wortart Präposition eingegangen (\autoref{sec:PraepDE}).
\autoref{sec:Grammatikalisierung} behandelt die Entwicklung der Präpositionen aus grammatikalisierungstheoretischer Sicht. 
Anschließend geht es darum, was aufgrund bisheriger Untersuchungen bereits über die Indexikalität der Genitiv- und Dativrektion gesagt werden kann (\autoref{sec:IndexikalitaetRektionskasus}). 

Im empirischen Teil wird zunächst das methodische Vorgehen erläutert (\autoref{cha:Methode}): 
In \autoref{sec:Konzipierung} und \autoref{sec:Fragebogen} geht es darum, wie der Onlinefragebogen konzipiert wurde und wie er aufgebaut ist. 
Anschließend wird beschrieben, wie die Daten erhoben und aufbereitet wurden (\autoref{sec:DatenerhebungundAufbereitung}) und wie die Antworten der Befragten auf offene Fragen im Assoziationsteil und im Akzeptabilitätstest für eine qualitative Inhaltsanalyse kategorisiert wurden (\autoref{sec:Kategorisierung}). 

In \autoref{cha:Ergebnisse} werden die Ergebnisse der Untersuchung präsentiert. 
\autoref{sec:Befragte} geht auf die soziodemografische Zusammensetzung der Befragten ein und beschreibt allgemeine Einstellungen gegenüber Sprache, die im Fragebogen erhoben wurden, etwa wie tolerant Befragte gegenüber sprachlicher Variation im Allgemeinen sind. 
Anschließend erfolgt die Auswertung der drei Hauptteile des Fragebogens: der freien Assoziationen, des Akzeptabilitätstests und des Produktionsexperiments. 
In \autoref{sec:ErgAssAes} und \autoref{sec:ErgAkz} wird erläutert, welche sozialen Bewertungen der Dativ- und Genitivrektion die Analyse der freien Assoziationen und der Ergebnisse des Akzeptabilitätstests offenlegt.  
Den Ergebnissen des Produktionsexperiments und damit der Verwendung der Varianten ist \autoref{sec:ErgProduktion} gewidmet. 

\autoref{cha:Disk} und \autoref{cha:Ausblick} bilden das Fazit des Buchs.
Dabei erfolgt in \autoref{cha:Disk} zunächst die Zusammenfassung und Diskussion der Ergebnisse zur Bewertung der Rektionskasus und des Erklärungspotenzials, das sich daraus für die Verwendung der Varianten ergibt. 
Abschließend gibt \autoref{cha:Ausblick} einen Ausblick auf Anknüpfungspunkte, die die vorliegende Untersuchung für zukünftige Studien bietet. 

Die Ergebnisse der Studie legen nahe, dass die Präpositionalkasus Dativ und Genitiv über eine starke indexikalische Verweiskraft verfügen, die bei der Verwendung der Varianten nutzbar gemacht wird.  
Mit der systematischen Erhebung von Werturteilen zu Dativ- und Genitivvarianten ausgewählter Präpositionen trägt die Untersuchung somit zu einem besseren Verständnis der Variation % Änderung Anfang und des Wandels % Änderung Ende
der präpositionalen Rektion im Deutschen bei.
