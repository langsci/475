\title{Bewertung und Variation der Präpositionalkasus im Deutschen}
\subtitle{Der Einfluss metapragmatischer Urteile auf die Rektion von Präpositionen}
\author{Annika Vieregge}
\BackBody{Die Studie untersucht den Einfluss der metapragmatischen Bewertung von Genitiv und Dativ auf die Nutzung der Kasus in Präpositionalphrasen.
Die Variation zwischen Genitiv- und Dativrektion betrifft insbesondere Präpositionen, die noch nicht vollständig grammatikalisiert sind.
Daher wurde das Phänomen bisher vor allem aus grammatikalisierungstheoretischer Sicht betrachtet.
Dies greift jedoch zu kurz -- vielmehr hat die metapragmatische Bewertung der Kasus entscheidenden Einfluss auf die Kasuswahl.
Genitiv und Dativ werden von Sprecher:innen sehr unterschiedlich bewertet:
Der Genitiv gilt als Prestigekasus, der Dativ wird mit geringer Bildung und Umgangssprachlichkeit verbunden.

Daher werden einerseits Sprachideologien zu Dativ und Genitiv genauer beleuchtet und andererseits der Einfluss der metapragmatischen Bewertung auf die Kasuswahl.
Hierfür wurden exemplarisch die ursprünglichen Genitivpräpositionen \textit{wegen} und \textit{während} sowie die ursprünglichen Dativpräpositionen \textit{dank} und \textit{gegenüber} untersucht.
Zusätzlich wurde die Primärpräposition \textit{seit} in die Studie aufgenommen.
Die metagpragmatische Bewertung der Kasus wurde mitihlfe eines Akzeptabilitätstest und Abfragen freier Assoziationen untersucht, der Einfluss auf die Kasuswahl mithilfe von Produktionsdaten.
An der umfangreichen Onlinestudie nahmen 400 Muttersprachler:innen des Deutschen teil.
Die Arbeit ist damit die erste, die die metapragmatische Bewertung von Genitiv und Dativ in den Mittelpunkt stellt, systematisch erhebt und im Rahmen der Sprachideologieforschung diskutiert.

In der Studie zeigt sich deutlich, dass den untersuchten Rektionsvarianten eine ganze Reihe unterschiedlicher indexikalischer Bedeutungen zugeschrieben werden:
Der Genitiv wird als formell angesehen und steht für hohe Bildung, gute Sprachkenntnisse, Arroganz, Professionalität und Verkrampftheit.
Der Dativ gilt als informell und steht für geringe Bildung, mangelnde Sprachkenntnisse und Schlampigkeit.
Die im Fragebogen erhobenen Produktionsdaten verdeutlichen den Einfluss dieser metapragmatischen Bewertungen und bestätigen das mangelnde Erklärungspotenzial der Grammatikalisierungstheorie:
Sowohl bei ursprünglichen Genitiv- als auch bei ursprünglichen Dativpräpositionen lässt sich eine Tendenz zum Genitiv erkennen, insbesondere im formell gehaltenen Produktionsteil.
Diese Ergebnisse sprechen dafür, dass die Variation der präpositionalen Rektion in hohem Maße von der metapragmatischen Bewertung der Kasus beeinflusst wird und die Varianten entsprechend ihrer sozialen Bedeutung genutzt werden.}
\renewcommand{\lsISBNdigital}{978-3-96110-497-0}
\renewcommand{\lsISBNhardcover}{978-3-98554-126-3}
\BookDOI{10.5281/zenodo.14608443}
\proofreader{Benjamin Brosig,
            Daniela Schroeder,
            Jakob Prange,
            Jean Nitzke,
            Katja Politt,
            Ludger Paschen,
            Maria Zielenbach,
            Patricia Cabredo,
            Rainer Schulze,
            Stefan Hartmann,
            Tabea Reiner,
            Tom Bossuyt}
\renewcommand{\lsID}{475}

\lsCoverTitleSizes{43pt}{15mm}% Font setting for the title page

\renewcommand{\lsSeries}{ogl}
\renewcommand{\lsSeriesNumber}{10}
